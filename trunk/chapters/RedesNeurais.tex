\chapter{Redes Neurais Artificiais}
\label{Cap:RedesNeurais}

Redes Neurais Artificiais (RNAs) são modelos computacionais de comprovado poderio. São usadas primordialmente para a abstração e o reconhecimento de padrões, bem como para aproximação de funções, inclusive as não lineares.

\section{Breve Histórico das RNAs}

Em geral, livros creditam o pontapé inicial no estudo das RNAs ao trabalho de~\cite{McCulloch1943}, em que uma máquina foi construida inspirada no funcionamento do cérebro humano, fazendo um paralelo entre células nervosas vivas e o processo eletrônico. As conclusões dessa pesquisa foram de extrema importância para a futura implementação computacional do neurônio formal, são elas:
\begin{itemize}
	\item A atividade do neurônio é \emph{tudo ou nada}
	\item A atividade de qualquer sinapse inibitória previne a excitação do neurônio naquele instante.
\end{itemize}

De acordo com~\cite{Cardon1994}, a primeira afirmação significa que o neurônio estará no estado ativado se a sua saída ultrapassar um valor limite, caso contrário, ficará no estado de repouso (este princípio originou a função limiar). Entende-se por estado ativado transmitir a saída (transmissão) a outros neurônios da rede. Já a segunda afirmação teve importância na construção do neurônio formal a partir do conceito de pesos.

