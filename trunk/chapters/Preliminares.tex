\chapter{Preliminares}
\label{Cap:Preliminares}

Uma~\emph{Rede Neural Artificial} (RNA) é um modelo computacional capaz de abstrair uma determinada informação dado um conjunto de exemplos dela. Esquematicamente, ela é representada na forma de um grafo~\cite{LIVRO:1996.9780201539219} e é composta por unidades chamadas de~\emph{neurônios}, cujo objetivo principal é emular o funcionamento de um neurônio real, mesmo que de forma grosseira~\cite{LIVRO:2003.9780203451519}. O~\emph{conhecimento} de uma rede neural artificial está no valor dos pesos que ligam seus neurônios, exceto em redes sem peso~\cite{ARTIGO:10.1108, ARTIGO:1959.1460326}, que não serão utilizadas nesse trabalho.

A larga utilização de redes neurais artificiais para a predição de valores em séries temporais~\cite{ARTIGO:1992.253659, ARTIGO:1992.253669} se explica porque elas são ideais para esse propósito~\cite{ARTIGO:1992.153489} e tem a capacidade de aproximar qualquer função, contínua ou não~\cite{LIVRO:2009.9788591020805}. Para aumentar a acurácia na previsão da rede, recomenda-se pré-processar a série temporal com o objetivo de atenuar o seu ruído~\cite{ARTIGO:2010.5345722}.

O uso de~\emph{onduletas}\footnote{Embora a maioria dos autores prefira usar o termo em inglês~\emph{wavelet}, neste trabalho usamos o termo na língua portuguesa já que este foi o padrão escolhido nos demais termos usados no trabalho} é descrito por~\cite{ARTIGO:1992.165591, ARTIGO:2001.938397} a fim de pré-processar séries temporais e, com isso, atenuar seu ruído. Onduletas são funções matemáticas que satisfazem certas condições e que são capazes de representar dados ou outras funções~\cite{ARTICLE:1995.388960} tanto no eixo de frequências quanto no de tempo. Dentre as tantas onduletas existentes, a~\emph{Transformada de Haar} foi a escolhida para este trabalho porque, de acordo com~\cite{ARTIGO:2016.Reddy2016}, a transformada de Haar é matematicamente simples, converge rápido e possui alta precisão.

\section{Nomenclatura}

Seja $y$ uma série temporal de $Y$ elementos e $y_i$ um elemento de $y$ tal que $0 < i < Y$. 

\section{Trabalhos relacionados}

% Explicar wavelet: http://ieeexplore.ieee.org/document/1704233/

%http://ieeexplore.ieee.org/document/165591/

%http://ieeexplore.ieee.org/document/938397/

%http://ieeexplore.ieee.org/document/7091522/

%http://ieeexplore.ieee.org/search/searchresult.jsp?queryText=neural%20network%20time%20series%20prediction%20wavelet&sortType=asc_p_Publication_Year

%http://ieeexplore.ieee.org/document/5345722/

%http://ieeexplore.ieee.org/search/searchresult.jsp?newsearch=true&queryText=time%20series%20denoising%20wavelets

% https://arxiv.org/abs/1010.4084

% História wavelet: http://www.iieta.org/sites/default/files/Journals/MMEP/Vol3_No2_2016/03.2_12.pdf

% http://ieeexplore.ieee.org/document/409519/
